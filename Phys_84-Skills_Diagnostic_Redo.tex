\documentclass[11pt,letter]{article}
\usepackage[margin=3cm]{geometry}

% ------------------------------------------------------------------------------

% Physics 84 --- Skills Diagnostic Redo

% This skills diagnostic was assigned as a part of Physics 84 (Quantum
% Information) at Harvey Mudd College during Spring semester 2015. It was given
% as a teaser the the first class meeting, then assigned as homework once that
% class had gotten going.

% ------------------------------------------------------------------------------

% AUTHORSHIP, COPYRIGHT and LICENSE

% Written in 2015 by Benjamin Johnson.

% To the extent possible under law, the author(s) have dedicated all copyright
% and related and neighboring rights to this source code to the public domain
% worldwide. This source code is distributed without any warranty.

% You should have received a copy of the CC0 Public Domain Dedication along with
% this source code. If not, see
% <http://creativecommons.org/publicdomain/zero/1.0/>.

% ------------------------------------------------------------------------------

%% Document Metadata

\usepackage{./hw_metadata}

% Course metadata
\setCourseDepartment{Physics}
\setCourseNumber{84}
\setCourseTitle{Quantum Information}
%\setCourseSection{1} % Omit when there is only one section

% Assignment metadata
\setAssignmentTitle{Skills Diagnostic Redo}
\setDate{2015-01-29}

% Author metadata
\setAuthor{} % Name of student completing the assignment.

% Acknowledgments (acks)
%\addAck{}

%% Packages and settings for titles, headers, and footers,
%% and for layout features like tables, images, and colore

\usepackage{./hw_layout}

% I've separated these out of the package so that the authorship and usage
% restrictions may be more easily adjusted from within the document.
\hypersetup{
  colorlinks={true},
  linkcolor={black},
  urlcolor={black},
  pdfauthor={\theAuthor{}},
  pdfcopyright={To the extent possible under law, the author(s) have waived
    all copyright and related or neighboring rights to this work.},
  pdflicenseurl={https://creativecommons.org/publicdomain/zero/1.0/}
}
\lfoot{\ccZeroBar}
\fancypagestyle{fancytitle}{%
  \renewcommand{\headrulewidth}{0pt}%
  \fancyhf{}%
  \fancyfoot[L]{\ccZeroBar}%
  \fancyfoot[C]{\sc \thepage\ of \pageref{LastPage}}%
}

% ------------------------------------------------------------------------------

%% Packages and settings for typesetting primary content

% This must come before hw_type due to a conflict between amsthm and fontspec
% Look inside the (meta)package for more details on all it provides
\usepackage{./hw_math}

% This need not bother with settings for math typesetting.
% This must come after hw_math due to a conflict between amsthm and fontspec
\usepackage{./hw_type}

% I'm experimenting with using Unicode math.
% This should come last so it can override previous settings
\usepackage[math-style=ISO]{unicode-math}
\setmathfont{texgyrepagella-math.otf}

% unicode-math changes the behavior of bold math, and therefore affects vectors.
% Now this method works as it should, so the previous kludge is unnecessary.
% This supercedes my old vector code.
\newcommand{\vvec}[1]{\vec{\mathbf{#1}}} % Arbitrary vector (bold, arrow)
\newcommand{\uvec}[1]{\hat{\mathbf{#1}}} % Unit vector (bold, hat)
\renewcommand{\vv}[1]{\vvec{#1}} % Alias for backward compatibility.
\renewcommand{\uv}[1]{\uvec{#1}} % Alias for backward compatibility.
\newcommand{\vmat}[1]{\vec{\mathbf{#1}}} % Arbitrary matrix (bold, arrow)
\newcommand{\umat}[1]{\hat{\mathbf{#1}}} % Unitary matrix (bold, hat)

% Define '\incr' for the increment operator, commonly refered to and typeset as
% Greek Capital Delta, but which in fact has its own Unicode code point and
% LaTeX operator --- the ISO standard. Appearance may vary significantly or not
% at all depending on typeface.
\newcommand*\incr{\mathop{}\!\mathbin{}∆}

% Per ISO 80000, true mathematical constants (e, i, π) are best typeset upright,
% not italic, just like units (m, kg, Å) and mathematical operations (sin, lim)

% An upright 'e' for Euler's number
\newcommand{\mathe}{\mathrm{e}}
% An upright 'i' for the imaginary unit
\newcommand{\mathi}{\mathrm{i}}
\renewcommand{\I}{\mathi}
% An upright pi for the mathematical constant
\newcommand{\mathpi}{\mathrm{\pi}}

% ------------------------------------------------------------------------------

% PACKAGES and MACROS for THIS FILE only

% This package provides pretty syntax highlighting and code formatting
% for a variety of programming languages.
%\newcommand\TestAppExists[3]{#2}
%\usepackage{minted}

% ------------------------------------------------------------------------------

\begin{document}

% Typeset the first page header
\maketitle
\thispagestyle{fancytitle}
\tabulateAcks

% ------------------------------------------------------------------------------

\begin{problem}[0.1 Eigenvalues and Eigenvectors]
  Find the eigenvalues and eigenvectors of the following matrix:
  \begin{align*}
    \begin{bmatrix} 2 & \mathi \\ -\mathi & 2 \end{bmatrix}.
  \end{align*}
\end{problem}

\begin{solution}

\end{solution}

\pagebreak % -------------------------------------------------------------------

\begin{problem}[0.2 Digital Logic Circuits]
  Design and draw a simple logic circuit that takes two classical bits as input
  and is described by the following truth table:
  \begin{table}[H]
    \centering
    \begin{tabular}{l c || r}
    A & B & Out \\
    \hline
    0 & 0 & 1 \\
    0 & 1 & 1 \\
    1 & 0 & 0 \\
    1 & 1 & 1
  \end{tabular}
\end{table}

\end{problem}

\begin{solution}

\end{solution}

\pagebreak % -------------------------------------------------------------------

\begin{problem}[0.3 Qubit States and Measurement]
    A qubit is prepared in the quantum state $\ket{\psi} =
    \frac{1}{\sqrt{3}}\ket{0} + \mathi\sqrt{\frac{2}{3}}\ket{1}$.
  \begin{enumerate}[(a)]
  \item What is the probability that an ideal projective measurement in the
    $\cbr{\ket{a},\ket{a_{\perp}}}$ basis will find the qubit in the state
    $\ket{a}$ if $\ket{a} = \ket{0}$?
  \item What if $\ket{a} = \frac{1}{\sqrt{2}}(\ket{0}+\ket{1})$?
  \end{enumerate}
\end{problem}

\begin{solution}
  % \begin{enumerate}[(a)]
  % \item
  % \item
  % \end{enumerate}
\end{solution}

\pagebreak % -------------------------------------------------------------------

\begin{problem}[0.4 Matrix Operations]
  Write down a matrix $\vmat{U}_{\chi}$ such that
  $\vmat{U}_{\chi} \begin{bmatrix} a \\ b\mathe^{\mathi\varphi} \end{bmatrix}
  = \begin{bmatrix} a \\ b\mathe^{\mathi\del{\varphi+\chi}} \end{bmatrix}$ for
  any real numbers $a$, $b$, and $\varphi$.  Is $\vmat{U}_{\chi}$ a unitary
  matrix?
\end{problem}

\begin{solution}

\end{solution}

% ------------------------------------------------------------------------------

\vfill

\hrule
\vspace{\baselineskip}

\begin{center}
  \ccZeroTextNotice{}
\end{center}

% ------------------------------------------------------------------------------

\end{document}

% ------------------------------------------------------------------------------
